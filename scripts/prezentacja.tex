\documentclass[aspectratio=169]{beamer}
\usetheme{metropolis}

\usepackage[utf8]{inputenc}
\usepackage[OT4]{polski}
\usepackage{flexisym}
\usepackage{parskip}


\title{Uczenie maszynowe do gry Pac-Man}
\subtitle{Laura Dymarczyk, Adrian Rupala}


\begin{document}
	\frame {
		\titlepage
	}

	\frame {
		\frametitle{Opis projektu}
		\fontsize{15pt}{7.2}\selectfont
		Projekt ten ma na celu stworzenie aplikacji, która od podstaw, sama nauczy się grać w grę Pac-Man. \\~
		
		Implementacja została wykonana w języku programowania Python 3 z wykorzystaniem bibliotek OpenAI Gym oraz Tensorflow.
	}
	
	\frame {
		\frametitle{Założenia projektu}
		\fontsize{15pt}{7.2}\selectfont
		Czyli slajd o marzeniach...

	}
	
	\frame {
		\frametitle{Wykorzystane technologie}
		\fontsize{15pt}{7.2}\selectfont
		Do stworzenia tego projektu wykorzystaliśmy następujące technologie...
		
	}

	\frame {
		\frametitle{Napotkane problemy}
		\fontsize{15pt}{7.2}\selectfont
		Dlaczego świat nas rani...
	
	}

	\frame {
		\frametitle{Proces uczenia się}
		\fontsize{15pt}{7.2}\selectfont
		Ziuuu ładny GIF jak Pac-Man sobie biega...
	
	}
	
	\frame {
		\frametitle{Efekt końcowy}
		\fontsize{15pt}{7.2}\selectfont
		Jak w prawie każdym anime, wszystko dobrze się skończyło...
		
	}
	
	\frame{
		\centering \Huge
		\emph{Dziękujemy za uwagę!}
	}
	
\end{document}
